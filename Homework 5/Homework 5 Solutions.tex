\documentclass[letterpaper]{article}
\usepackage{textcomp}
\usepackage{amsmath}
\usepackage{IEEEtrantools}
\usepackage[T1]{fontenc}
\usepackage{lmodern}
\usepackage{verbatim}
\usepackage{graphicx}
\usepackage{enumerate}

\pagestyle{headings}
\title{PHYS 2030 \\Homework 5 Solutions.}
\usepackage[pdftex]{hyperref}
\begin{document}

\maketitle
\begin{enumerate}
\item
\begin{enumerate}[a.]
\item -
\item Let us denote $v=\dfrac{dx}{dt}$, then the undriven Duffing oscillator can be rewritten as the following system of ODEs.
\begin{IEEEeqnarray*}{rCl}
& \dfrac{dx}{dt} & = v\\
& \dfrac{dv}{dt} & = -\alpha x-\beta x^3-\delta v
\end{IEEEeqnarray*}
This system cannot be represented in matrix form, because of its non-linearity, due to $-\beta x^3$ term.
\item To find the equilibrium points we need to find such $v_0=v(t_0)$ and $x_0=x(t_0)$ for which $\dfrac{dx}{dt}\big|_{t,x_0,v_0}=0$ and $\dfrac{dv}{dt}\big|_{t,x_0,v_0}=0$. These $v_0$ and $x_0$ represent solutions that are constant for all time $t$. 

For our system of ODEs we have
\begin{IEEEeqnarray*}{rCl}
& \dfrac{dx}{dt} & =v=0\\
& \dfrac{dv}{dt} & =-\alpha x-\beta x^3-\delta v=0
\end{IEEEeqnarray*}
Simplifying, we get
\begin{IEEEeqnarray*}{rCl}
v & = & 0\\
x(\alpha+\beta x^2) & = & 0 
\end{IEEEeqnarray*}

\item In the linear ODE there is one nonlinear term, $-\beta x^3$.
\end{enumerate}
\end{enumerate}
\end{document}