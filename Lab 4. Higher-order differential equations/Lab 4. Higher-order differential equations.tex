\documentclass[letterpaper]{article}
\usepackage{textcomp}
\usepackage{amsmath}
%\usepackage{IEEEtrantools}
%\usepackage[T1]{fontenc}
\usepackage{lmodern}
\usepackage{verbatim}
\usepackage{graphicx}

\pagestyle{headings}
\title{PHYS 2030 \\Lab 4}
\usepackage[pdftex]{hyperref}
\begin{document}

\maketitle
Finish as many parts of the problem set as possible (and do not worry if you cannot finish all of them, since there is a lot to do). The problems are similar to the ones in Home work assignment 4.

Suggestions: 
\begin{itemize}
\item Put lots of comments in your code. You might use some parts of the code for your next assignments (or projects), and one thing you do not want to do is to spend time on trying to recall after couple of months what does your code do and why does it do it in that particular way. In addition, commenting will help you track down any logical errors, which are very hard to find in general.
\item  Use meaningful names for your variables (even if they turn out to get somewhat long). The reason for this is the same as above.
\item Do not just comment the particular lines of your code, but you can break your code into logical sections and give clear explanation of what is this section for.
\end{itemize}

\emph{You do not have to hand in anything and this problem set is not going to be graded.} 
\begin{enumerate}
\item Implement 2nd-order Runge-Kutta method for $\dfrac{d \textbf{y(t)}}{dt}=f(t, \textbf{y}(t))$, which is the following
\begin{equation*}
\textbf{y}(t+\Delta t)=\textbf{y}(t)+\Delta t \cdot f \left(t+\dfrac{\Delta t}{2},\textbf{y}(t)+\dfrac{\Delta t}{2} \cdot f(t,\textbf{y}(t))\right)
\end{equation*}
\item Now modify the 2nd-order Runge-Kutta method function to solve one body 1D mechanics problems. I.e., of the form 
\begin{equation*}
\dfrac{d^2x(t)}{dt^2}=\dfrac{(F_{net})_{x}}{m}
\end{equation*}
This function must have two inputs for initial conditions: $x(t_0)$ and \mbox{$\dfrac{dx(t_0)}{dt}=v(t_0)$}. The input function is $\dfrac{(F_{net})_{x}}{m}$.
\item Test your new function by solving the problem of simple harmonic motion (SHO) modeled by a body of mass $m$ on a spring with spring constant $k$. The equation of motion is 
\begin{equation*}
\dfrac{d^2x(t)}{dt^2}=-\dfrac{k}{m}x
\end{equation*}
\item Pick your own initial conditions and perform the numerical computation for a set of time steps. 
\item On one plot show the position-vs-time graphs for all time steps. Make sure it is easy to distinguish different computed data sets. 
\item Show on one plot both the analytic solution and the data corresponding to the smallest element of your set of time steps. The analytical solution is given by
\begin{equation*}
x(t)=A \sin (\omega t+\phi_0),
\end{equation*}
where $A$ is the oscillation amplitude, $\omega$ is the angular frequency and $\phi_0$ is the initial phase. Both the amplitude and the initial phase can be derived from the initial conditions. The initial phase is easy to find and for the amplitude use energy conservation.
\item Solve the SHO problem with \verb|ode45| with the same initial conditions as before. (Go to MatLab \emph{Help} file and read about syntax for \verb|ode45|.)
\item Show on separate graphs the computed data and associated phase space. Comment on shape of the phase space.
\end{enumerate}
\end{document}